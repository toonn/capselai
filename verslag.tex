%%%%%%%%%%%%%%%%%%%%%%%%%%%%%%%%%%%%%%%%%
% Journal Article
% LaTeX Template
% Version 1.3 (9/9/13)
%
% This template has been downloaded from:
% http://www.LaTeXTemplates.com
%
% Original author:
% Frits Wenneker (http://www.howtotex.com)
%
% License:
% CC BY-NC-SA 3.0 (http://creativecommons.org/licenses/by-nc-sa/3.0/)
%
%%%%%%%%%%%%%%%%%%%%%%%%%%%%%%%%%%%%%%%%%

%----------------------------------------------------------------------------------------
%	PACKAGES AND OTHER DOCUMENT CONFIGURATIONS
%----------------------------------------------------------------------------------------

\documentclass[twoside]{article}

\usepackage{lipsum} % Package to generate dummy text throughout this template

\usepackage[sc]{mathpazo} % Use the Palatino font
\usepackage[T1]{fontenc} % Use 8-bit encoding that has 256 glyphs
\linespread{1.05} % Line spacing - Palatino needs more space between lines
\usepackage{microtype} % Slightly tweak font spacing for aesthetics

\usepackage[hmarginratio=1:1,top=32mm,columnsep=20pt]{geometry} % Document margins
\usepackage{multicol} % Used for the two-column layout of the document
\usepackage[hang, small,labelfont=bf,up,textfont=it,up]{caption} % Custom captions under/above floats in tables or figures
\usepackage{booktabs} % Horizontal rules in tables
\usepackage{float} % Required for tables and figures in the multi-column environment - they need to be placed in specific locations with the [H] (e.g. \begin{table}[H])
\usepackage{hyperref} % For hyperlinks in the PDF

\usepackage{lettrine} % The lettrine is the first enlarged letter at the beginning of the text
\usepackage{paralist} % Used for the compactitem environment which makes bullet points with less space between them

\usepackage{abstract} % Allows abstract customization
\renewcommand{\abstractnamefont}{\normalfont\bfseries} % Set the "Abstract" text to bold
\renewcommand{\abstracttextfont}{\normalfont\small\itshape} % Set the abstract itself to small italic text

\usepackage{titlesec} % Allows customization of titles
\renewcommand\thesection{\Roman{section}} % Roman numerals for the sections
\renewcommand\thesubsection{\Roman{subsection}} % Roman numerals for subsections
\titleformat{\section}[block]{\large\scshape\centering}{\thesection.}{1em}{} % Change the look of the section titles
\titleformat{\subsection}[block]{\large}{\thesubsection.}{1em}{} % Change the look of the section titles

%\usepackage{fancyhdr} % Headers and footers
%\pagestyle{fancy} % All pages have headers and footers
%\fancyhead{} % Blank out the default header
%\fancyfoot{} % Blank out the default footer
%\fancyhead[C]{Running title $\bullet$ November 2012 $\bullet$ Vol. XXI, No. 1} % Custom header text
%\fancyfoot[RO,LE]{\thepage} % Custom footer text

%----------------------------------------------------------------------------------------
%	TITLE SECTION
%----------------------------------------------------------------------------------------

\title{\vspace{-15mm}\fontsize{24pt}{10pt}\selectfont\textbf{Gebruik van compressie voor gegevensclustering}} % Article title

\author{
\large
\textsc{Toon Nolten, Joren Verspeurt} \\% Your name
\normalsize KULeuven \\ % Your institution
\normalsize r0258417, r0258654
\vspace{-5mm}
}
\date{}

%----------------------------------------------------------------------------------------

\begin{document}

\maketitle % Insert title

%\thispagestyle{fancy} % All pages have headers and footers

%----------------------------------------------------------------------------------------
%	ABSTRACT
%----------------------------------------------------------------------------------------

\begin{abstract}

\emph{Clusteren} is het groeperen van data in een reeks inputgegevens. 
Het kan moeilijk zijn om een dergelijke groepering uit te voeren zonder voorafgaand begrip
van de gegevens die bekeken worden. 
Dit is problematisch als men op zoek is naar een manier om die gegevens beter te begrijpen 
aan de hand van de verbanden tussen de verschillende data in de reeks.
Compressie kan dan gebruikt worden om een afstand tussen verschillende inputs te berekenen
op een algemene manier die gebruikt kan worden voor zeer verscheidene bronnen zoals tekst,
afbeeldingen, video, ...
Dit is gebaseerd op het principe dat compressie gegevens in het beste geval reduceert tot 
de absolute hoeveelheid informatie aanwezig in die gegevens.
Als meerdere data samen gecomprimeerd worden kunnen dus redundante delen ge\"elimineerd 
worden. 
De hoeveelheid gemeenschappelijke informatie kan dan gebruikt worden als een afstandsmaat.
Hier wordt de bruikbaarheid van compressie voor clustering en de invloed van verschillende 
compressiealgoritmes onderzocht.

\end{abstract}

%----------------------------------------------------------------------------------------
%	ARTICLE CONTENTS
%----------------------------------------------------------------------------------------

\begin{multicols}{2} % Two-column layout throughout the main article text

\section{Introductie}

\lettrine[nindent=0em,lines=3]{C} lusteren van gegevens heeft toepassingen in diverse 
situaties. 
Het kan gebruikt worden om gegevens te sorteren, zoals bijvoorbeeld in het geval men
een collectie afbeeldingen heeft waarvan men vermoedt dat deze bestaat uit een 
samenstelling van een aantal collecties afbeeldingen waarbij elke collectie afbeeldingen
bevat van een zelfde onderwerp maar dit onderwerp verschilt tussen de collecties.
Ook zou men tekst kunnen sorteren op de taal of het onderwerp.
Het kan ook nuttig zijn om beter te begrijpen welke gegevens meer gerelateerd zijn of 
omgekeerd welke gegevens ver verwijderd liggen van de rest (zogenaamde \emph{outlier 
detection}).
Eens men een clustering heeft kan deze ook gebruikt worden om aan categorisatie te doen.
Nieuwe gegevens kunnen dan ingedeeld worden bij de cluster waar ze het minst ver van 
verwijderd zijn.
Compressie biedt hier een heel algemene manier om aan clustering te doen omdat men door
vergelijking van de gecomprimeerde groottes van de verschillende gegevens afzonderlijk 
en de gecomprimeerde groottes van combinaties van gegevens afstanden kan afleiden tussen
de gegevens, wat dan een vorm van hi\"erarchisch (of ook wel agglomeratief) clusteren
mogelijk maakt. Deze afstanden kunnen op verschillende manieren gedefini\"eerd worden,
wat ook verschillende vormen van clusters tot gevolg heeft.

%------------------------------------------------

\section{Methoden}

Er zijn veel mogelijke manieren om de effectiviteit van het gebruik van compressie voor
clustering en het effect van de sterkte van de compressie op deze effectiviteit te
onderzoeken.
\\
De methoden die wij gebruikt hebben en hier bespreken hebben we zelf enkel getest op tekst
maar zijn mogelijk toepasbaar op vele verschillende soorten gegevens.
\\
In de eerste uitgevoerde experimenten werd gebruik gemaakt van courant beschikbare 
compressieprogramma's om op een automatische manier afzonderlijke tekstbestanden en 
combinaties van tekstbestanden te comprimeren. De groottes van deze bestanden werden dan
bijgehouden in een tekstbestand dat dan later ingelezen kon worden door een zelf geschreven
clusterprogramma.
%%Onderstaande zin is een beetje vreemd.
Uit interesse voor de toepassing van het declaratief programmeerparadigma op dit probleem
werd aanvankelijk gekozen voor een Prolog-programma voor de clustering.
Dit leidt tot een vrij compact en flexibel programma, maar de complexiteit in functie van 
het aantal items dat geclusterd moest worden bleek vrij slecht.
\begin{compactitem}

\end{compactitem}


%------------------------------------------------

\section{Resultaten}

\begin{table}[H]
\caption{Example table}
\centering
\begin{tabular}{llr}
\toprule
\multicolumn{2}{c}{Name} \\
\cmidrule(r){1-2}
First name & Last Name & Grade \\
\midrule
John & Doe & $7.5$ \\
Richard & Miles & $2$ \\
\bottomrule
\end{tabular}
\end{table}

\lipsum[5] % Dummy text

\begin{equation}
\label{eq:emc}
e = mc^2
\end{equation}

\lipsum[6] % Dummy text

%------------------------------------------------

\section{Discussion}

\subsection{Subsection One}

\lipsum[7] % Dummy text

\subsection{Subsection Two}

\lipsum[8] % Dummy text

%----------------------------------------------------------------------------------------
%	REFERENCE LIST
%----------------------------------------------------------------------------------------

\begin{thebibliography}{99} % Bibliography - this is intentionally simple in this template

\bibitem[Figueredo and Wolf, 2009]{Figueredo:2009dg}
Figueredo, A.~J. and Wolf, P. S.~A. (2009).
\newblock Assortative pairing and life history strategy - a cross-cultural
  study.
\newblock {\em Human Nature}, 20:317--330.
 
\end{thebibliography}

%----------------------------------------------------------------------------------------

\end{multicols}

\end{document}
